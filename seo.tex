\documentclass[ngerman, 12pt, a4paper]{scrartcl}
\usepackage[ngerman]{babel}

\usepackage[T1]{fontenc}
\usepackage[utf8]{inputenc}
\usepackage{xspace}
\usepackage{times}
\usepackage{graphicx}
\DeclareGraphicsExtensions{.pdf,.png,.jpg}

\usepackage[style=numeric, backend=bibtex]{biblatex}
\usepackage[babel,german=guillemets]{csquotes}
\bibliography{seo.bib}

\title{Wie komme ich bei Google\newline auf Seite eins?}
\subtitle{Eine Anleitung zur Suchmaschinenoptimierung}
\author{Thomas Rega}
\date{\today}
\parindent 0pt

\usepackage{hyperref}

\hypersetup{
    pdfstartview={FitH},
    pdftitle={Wie komme ich bei Google auf Seite eins? Eine Anleitung zur Suchmaschinenoptimierung},
    pdfauthor={Thomas Rega},
    pdfsubject={Wie komme ich bei Google auf Seite eins? Eine Anleitung zur Suchmaschinenoptimierung},
    pdfcreator={Thomas Rega},
    pdfproducer={Thomas Rega},
    pdfkeywords={bei google auf seite eins kommen} {google optimierung} {seo},
    pdfnewwindow=true,
    colorlinks=true,
    linkcolor=black,
    citecolor=black,
    filecolor=black,
    urlcolor=black
}


\begin{document}
\maketitle

\newpage

Der Algorithmus, welcher von Google zur Auflistung von Suchergebnissen benutzt wird ist mindestens
so komplex wie ``Sagen umwoben'' - vor allem ist er jedoch eines: geheim.
\newline

Dies ist der Grund, warum die Aussagen bezüglich des Rankings bei Google nur als Mutmaßungen und
Spekulationen eingestuft werden können. Die folgenden Informationen beruhen auf eigenen Erfahrungen.
Es besteht kein Anspruch auf Vollständigkeit und Exaktheit.
\newline

Google selbst bietet Hilfe zur Suchmaschinenoptimierung unter \cite{seo} an.

\newpage

\section*{Inhalt}

Suchmaschinen brauchen Futter. Suchmaschine brauchen Inhalt, den sie indizieren, auf den sie
verweisen können. Stelle ihnen diesen Inhalt zur Verfügung. Zusätzlich lassen sich auswertbare
Informationen in sog. Meta Angaben (``Meta Tags'') unterbringen. Die Indizierungsprogramme
(sog. ``Robots'' oder ``Crawler'') der Suchmaschinen registrieren die Meta Tags einer Seite und nutzen
diese zur Indizierung. Weitere Informationen zu Meta Tags unter: \cite{meta_tags} und
\cite{meta_tags_selfhtml}

\paragraph{Quellen}
Der Inhalt der Seite sollte ausschließlich von Dir selbst erstellt worden sein, damit
sind ausdrücklich auch die Bilder gemeint. Falls es nötig sein sollte externe Quellen zu bemühen
und/oder auf externe Quellen zu verweisen, so sind dies stets ersichtlich zu gestalten und ggf.
vorhandene Copyrights und Nutzungsbedingungen zu beachten. Vermeide auf jeden Fall ``zu Guttenbergen''
(copy-and-paste). Sollten solche Plagiate erkannt werden, so führt dies zu einer Herabstufung im Ranking.

\paragraph{Description}
Der `description' Meta Tag sollte in etwa 100 - 120 Zeichen lang sein und zum Anklicken einladen.
Eine gute Kurz-Beschreibung animiert den Nutzer zum Besuch Deiner Webseite. Fasse den Inhalt/Absicht
der Seite in zwei bis drei Sätzen zusammen.
\newline Siehe auch: \cite{meta_tags}

\paragraph{Keywords}
Der `keyword' Meta Tag sollte die Schlüsselwörter beinhalten, unter welchen man bei der Eingabe in
einer Suchmaschine gefunden werden möchte. Es nützt allerdings nichts, wenn der `keyword' Tag die
einzige Stelle ist, an der diese Schlüsselwörter auftauchen. Benutze Webmaster Tools um zu kontrollieren,
welche Keywords vom Robot der jeweilige Suchmaschine als solche bemerkt und erkannt werden.
\newline Siehe auch: \cite{google_webmaster}, \cite{meta_tags}

\paragraph{Sitemap}
Google empfiehlt die Hinterlegung einer Sitemap der Seite. Die Sitemap unterstützt den Google Bot
bei der Indizierung der Seite. Unter \cite{sitemap} werden Informationen zur Erstellung einer Sitemap
zur Verfügung gestellt. Neben URLs können in einer Sitemap auch Informationen zur letzten Aktualisierung
der Seite hinterlegt werden. Per Google Webmaster Tools hat der Webmaster der Seite die Möglichkeit
auf eine Aktualisierung der Sitemap bzw. deren Erstellung hinzuweisen, in dem er die Sitemap dort
``einreicht'' und so einen Crawling Vorgang anstößt.

\paragraph{Aktualität}
Aktuelle Informationen lassen sich mit Hilfe von Blog Einträgen in die Seite einfügen.
Wenn die Seite "lebt", d.h. wenn dort kontinuierlich neue Artikel / Informationen hinterlegt werden,
so wird sich das auf keinen Fall negativ auf das Ranking bei Google auswirken - im Gegenteil.

\paragraph{Erfahrungsberichte}
Authentische Bewertungen (auch "Testimonials" genannt) wirken sich ebenfalls günstig auf die
Gesamtbewertung der Seite aus. Siehe auch:  \cite{testimonial}


\section*{Back-Links}
Es existiert das Gerücht, dass Google zehn Top Seiten (z.Bsp. "The New York Times") mit dem Rank 1
versehen hat. Diese Seiten genießen das höchste Ansehen (Rank 1) und bilden eine Art Wurzel. Von
diesen Seiten aus wird auf Inhalte auf anderen Webseiten gezeigt, diese Webseiten erhalten dann den
Rank 2 (da eine Rank 1 Seite auf diese verweist) - wird von der Seite mit Rank 2 auf eine weitere
Seite verlinkt, so erhält diese Rank 3 usw. - so landet man schließlich irgendwann bei
Rank 3456122099, d.h. man ist sehr weit weg von der New York Times.

Verlinkungen auf die eigene Seite (sog. ``Back Links'') sind wichtig, da Google davon aus zu gehen
scheint, dass auf wertvolle Informationen auch von anderer Seite aus verwiesen wird. Bei den
Verlinkungen, welche auf die eigene Seite zeigen, sollte man darauf achten, dass diese von
``hochwertigen'' Seiten stammen. D.h. von Seiten, welche schon ``einiges'' an Prestige (Popularität)
mitbringen. Verlinkungen von ``niederwertigen'' Seiten (Seiten, welche dafür bekannt sind einzig und
allein als ``Linkfarm'' \cite{link_farm} zu gelten oder Seiten mit fragwürdigen Inhalten) wirken
sich dagegen negativ auf das Ranking aus.

\paragraph*{Social Networks}
Beiträge auf sozialen Netzwerken wie Facebook, G+, Twitter, Instagram, Xing, LinkedIn, Pinterest usw.
können dazu benutzt werden um Back Links zu erzeugen und die Popularität der eigenen Seite weiter
zu steigern. Meinungsäußerungen, Produktvorstellungen und sonstige Informationen, welche sich über
soziale Netzwerken verteilen, erscheinen zunehmend in den Suchergebnissen. Diese Entwicklung wird
wohl in naher Zukunft auch nicht abschwächen. Auf G+ gibt es das Angebot von Communities, welche
sich mit bestimmten Themen (Programmierung, Sport, Fitness, Wellness etc.) austauschen. Auf Xing,
Facebook etc. gibt es das Konzept der Gruppen. Hier bietet sich eine hervorragende Möglichkeit sich,
durch beispielsweise Beantwortung von Fragen oder das Verfassen von Fachartikeln, auf dem jeweiligen
Gebiet als Experte zu etablieren und dabei regelmäßig Verlinkungen auf die eigene Seite zu verteilen.
Die Verlinkungen sollten nicht ausschließlich auf die Landing Page zeigen, sondern durchaus direkt
auf die Artikel, welche auf der eigenen Seiten hinterlegt wurden. So lässt auf der einen Seite ein
Mehrwert für die Nutzer und Mitglieder der Communities und auf der anderen Seite Traffic für die
eigene Seite generieren.


\section*{Technik}
\paragraph*{Web Standards}
Der Web Standard ist zwar heute (07/2013) noch nicht in seiner finalen Form verabschiedet, es gibt
jedoch keinen Grund noch immer am alten Standard HTML4 fest zu halten. Jeder aktuelle Browser
unterstützt umfassend (d.h. noch nicht den kompletten Standard) HTML5 Elemente. Siehe auch: \cite{html5}

Eine der Neuerungen, die HTML5 mit sich bringt, sind neue Elemente zur semantischen Strukturierung
der Seite. Die Robots der Suchmaschinen werten die semantischen Struktur Elemente aus. Insgesamt
scheint es, als ob die Benutzung moderner Web Standards von den Suchmaschinen honoriert wird. Weiter
gehende Informationen auch zum Thema ``Semantisches Web'' unter: \cite{html5}, \cite{semantisches_web}.

\paragraph*{Kurze Ladezeiten}
Schnelle Auslieferungen von Inhalten und damit kurze Wartezeiten für den Nutzer werden honoriert.
Mit den Chrome Entwickler Tools lassen sich die Ladezeiten der einzelnen Bestandteile der Webseite
auf die Millisekunde genau nachvollziehen. Alles, was länger als eine Sekunde dauert, ist im Grunde nicht
akzeptabel. Wer hier Inhalte sparen kann, spart automatisch Ladezeit. Bilder sollten komprimiert,
Javascript und CSS Dateien zusammengefasst und verdichtet werden. Hierzu gibt es Tools - gute Web
Entwicklung Frameworks kommen mit Mechanismen, welche diese Aufgaben automatisch erledigen.\newline

Empfehlenswert ist der Einsatz von \texttt{mod\_pagespeed}. Dabei handelt es sich um ein Modul für
den Apache Webserver, welches die Konsolidierung und Komprimierung von Inhalten übernimmt.
\cite{pagespeed}\newline

Lesenswert sind auch die Artikel zum Thema ``Kurze Ladezeiten'' unter \cite{speed}. Zusammenfassend
lässt sich sagen, dass sich kurze Ladezeiten sowohl auf die User Experience als auch auf das Ranking
bei Google positiv auswirken.

\paragraph*{Sprechende URLs}
Die Zeiten von kryptischen URLs wie ``?id=42\&view=false'' sind vorbei. Nutzer wie
auch Suchmaschinen mögen ``sprechende URLs''. Solche URLs beschreiben schon in der URL an sich den
Inhalt der Seite. Sie enthalten Menschen lesbare Wortgruppen anstatt technischer Kürzel. \cite{speaking_urls}

\paragraph*{Analytics}
Wo gibt es Ausstiegspunkte, bei deren Erreichen fast alle Nutzer die Seite verlassen? Was suchen
die Nutzer auf meiner Seite? Woher kommen sie? Durch welchen Suchbegriff bei welcher Suchmaschine
sind die Nutzer auf meine Seite aufmerksam geworden? Welche Unterseite wird besonders häufig
aufgerufen? Wie viele der Nutzer waren schon auf der Seite und wie viele sind neu? Welche Browser
werden benutzt? Solche Fragen lassen sich mittels Google Analytics sehr gut nachvollziehen. Das Ganze kann
noch mit diversen Benutzer spezifischen Filtern und Abfragen versehen werden. Dieses Angebot von
Google kostet kein Geld - man bezahlt mit seinen Daten. \cite{analytics}

\newpage
\printbibliography[title=Ressourcen]
\end{document}